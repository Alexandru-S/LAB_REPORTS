\documentclass{article}

\usepackage{lipsum}
\usepackage[margin=1in,includefoot]{geometry}
\usepackage{graphicx}
\usepackage{float}
\usepackage[hidelinks]{hyperref}
\usepackage{amsmath}
\usepackage{amssymb}
\usepackage{color}


\usepackage[usenames,dvipsnames]{xcolor}
\usepackage{listings}







% Header and Footer Stuff
\usepackage{fancyhdr}
\pagestyle{fancy}
\fancyhead{}
\fancyfoot{}
\fancyfoot[R]{\thepage}
\renewcommand{\headrulewidth}{0pt}
\renewcommand{\footrulewidth}{0pt}


\definecolor{dkgreen}{rgb}{0,0.6,0}
\definecolor{gray}{rgb}{0.5,0.5,0.5}
\definecolor{mauve}{rgb}{0.58,0,0.82}

\lstset{
  language=VHDL,
  aboveskip=3mm,
  belowskip=3mm,
  showstringspaces=false,
  columns=flexible,
  basicstyle={\small\ttfamily},
  numbers=none,
  numberstyle=\tiny\color{gray},
  keywordstyle=\color{blue},
  commentstyle=\color{dkgreen},
  stringstyle=\color{mauve},
  breaklines=true,
  breakatwhitespace=true,
  tabsize=3
}


\begin{document}

\begin{titlepage}
	\begin{center}
	\begin{align*}
	\includegraphics[height=1.75in]{logo.png}
	\end{align*}


	
	\line(1,0){300}\\
	[0.25in]
	\huge{\bfseries Reflective Diary}\\
	[2mm]
	\line(1,0){200}\\
	[1.5cm]
	\textsc{\LARGE Case Study}\\
	[0.75cm]
	\textsc{\Large 4E1 Project Management}\\
	[7cm]	
	\end{center}
	
	
	
	\begin{flushright}
	\textsc{\large Alexandru Sulea\\
	D Stream\\
	\#12315152\\
	8 April 2016\\}
	\end{flushright}
	
\end{titlepage}
%Table of Contents Stuff%
\tableofcontents
%\listoffigures
%\addcontentsline{toc}{section}{List of Figures}
\listoftables
\addcontentsline{toc}{section}{List of Tables}


\thispagestyle{empty}
\cleardoublepage
\pagenumbering{arabic}
\setcounter{page}{1}

\pagebreak
\section{Simulation C}
Initially the simulation was conducted with the primary goal of delivering the product on time while maintaining the costs under budget. This solution was found to have a maximum score of 61.0 as the staff would constantly get confused and frustrated due to not knowing the situation. Outsourcing was also kept at a minimum due to the staff getting to do all the work.The stand up, peer-review and tutorial were also kept low to maximize the time the team spends working. The team size was kept at a minimum and increased to 3 only on week 4 due to the higher budget. Finishing the project on the revised deadline was not possible and the project could only be finished by week 16. The solution worked but only achieved minimal success. \\
This method was also vulnerable to low team morale and problems, if any of these problems arose the team could not perform to the best of their abilities and would finish the project late.
\\
The approach was then changed to a more rapid development cycle. This was initially a failure till the team deadline was changed to 8 weeks and the team size drastically increased from 2 to 7 team members. This timeline was chosen due to the management timeline for the simulation itself changing from 16 to 12 weeks on week 4. The team was not expected to finish the development cycle by week 8 but they would work harder thinking the deadline was shorter than it really was. Thus the team would be 3 weeks late according to the team deadline but would actually deliver the product 1 week ahead of the management deadline. To make sure that the team achieved the management deadline , the quality of the product goal was also increased to the next level, thus even if there was a problem with the team product goal, the management product goal would still be achieved.\\
The team struggled to achieve this goal initially thus outsourcing was used extensively from week one and overtime was encouraged. \\

This strategy further helped to increase the final score and have the team deliver the product ahead of schedule. \\
At this point the stand-ups and one to one meeting as well as the reviews were still being carried out on a weekly or by-weekly manner. This repeated process was found to only help at the start and would only cost time further in the development. Thus to further help the team achieve their objective the peer-review and one to one tutoring were kept to a minimum and only for the beginning of the project for 2 weeks.\\
The stand up was kept for 6 weeks due to it contributing to the teams morale and productivity.\\
The result of these changes was that the production cycle was drastically sped up but the project itself went up to 50 percent over budget. Although the project was completed over budget the fact that it was completed ahead of the revised schedule meant that it was seen as more of a success than a project finished late but under budget.





\pagebreak


\section{Simulation D}
Simulation D was somewhat more complicated and harder to score then simulation C, this is due to the fact that simulation objective for the quality of the finished product was already at maximum. 
For this simulation I initially attempted the same procedure as before, this however had a minimal effect. \\
Noticing how each product is different and therefore each product development cycle will also be unique to that product.\\
Initially the procedure was constructed to be budget efficient. This procedure however was not very effective when it came to the score. Thus once again when faced with a choice between delivering late and under budget or over budget and before the deadline it seems that the later is preferred. \\
Reusing lessons learned from the past simulation,  the team deadline was altered so that it was much shorter than the management deadline. This was so that the team would work harder but also so that it gave the project a safety net, thus even if a problem was discovered and it delayed the project it still wasnt enough of a delay to compromise the project.\\
Next the optimum team size had to be found. For this part a lot of trial and error was conducted. The most optimal team size was found to be 4 people at a medium high skill level.
For this project development cycle the outsourcing was kept at the none setting. This setting was found to be the most effective. Given the high skill demand of the project it was demanded that the team needed to know everything about the project. Thus minimizing the outsourcing increased the team skill and the team productivity.\\
One to one coaching was only set to 1 meeting, enough to keep the team focused but not too much to affect productivity.\\
Daily stand up and status review were also kept to a minimum. That is to keep the team morale up and productivity steady while also conserving time and money.\\
Lastly overtime was encouraged to increase productivity just as in all previous simulations. Overtime was found to increase productivity but at the same time also increase the probability of stress experienced by the team.\\
All things compared, this score was lower than the previous one. The reason for this score being significantly diminished was that there was far less flexibility on how the project could be conducted due to the project specs on the product being at the highest level. Building  a prototype was also not an option given the low budget and short time, coupled with the high quality demand in the product. Thus building a prototype would have many disadvantages and few if any advantages. Therefore no prototypes were built during the product development process.

\section{Simulation E}


The requirements for simulation E were closely realted to those of simulation D. Thus since every project process is unique to the project specifications, and if the project specifications are the same then the same process can be used.\\
Usign the same process from simulation D the highest score was achieved. This was then started with determining the optimum amount of employees. This again was seen to be 4 employees. Different permutation were atempted where the number of employees were significantyl decreased the closer the project came to completion. This however resulted in reduced scores. This can be due to the fact that because there were less employees present for the final days of the project, extra work was not done, thus resulting in a lower score.\\
The employees were again encouraged to take overtime in order to increase productivity. In order to offset the stress brought on by the overtime and short deadlines the employee daily stand up, status review and one to one coaching were kept active but to a minimum for the duration of the project production cycle. \\
A prototype was also not constructed during production, as previous attempts showed that construction a prototype will cost time and money at no benefit to the project. If a prototype is not specifically requested for , it will only become a hinderence to finishing the product on time.\\
Due to the fact that the product quality was again set to max by management , the only other option was to again decrease the team deadline to ten weeks ,  two less than the management team so that the team would work harder to finish the product ahead of time. If there were any problems encountered during the product phase, any delay encountered would not be substantial enough that it would prevent the product from being completed by the management deadline.\\


 







	
\end{document}