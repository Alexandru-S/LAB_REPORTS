\documentclass{article}

\usepackage{lipsum}
\usepackage[margin=1in,includefoot]{geometry}
\usepackage{graphicx}
\usepackage{float}
\usepackage[hidelinks]{hyperref}
\usepackage{amsmath}
\usepackage{amssymb}
\usepackage{color}


\usepackage[usenames,dvipsnames]{xcolor}
\usepackage{listings}







% Header and Footer Stuff
\usepackage{fancyhdr}
\pagestyle{fancy}
\fancyhead{}
\fancyfoot{}
\fancyfoot[R]{\thepage}
\renewcommand{\headrulewidth}{0pt}
\renewcommand{\footrulewidth}{0pt}


\definecolor{dkgreen}{rgb}{0,0.6,0}
\definecolor{gray}{rgb}{0.5,0.5,0.5}
\definecolor{mauve}{rgb}{0.58,0,0.82}

\lstset{
  language=VHDL,
  aboveskip=3mm,
  belowskip=3mm,
  showstringspaces=false,
  columns=flexible,
  basicstyle={\small\ttfamily},
  numbers=none,
  numberstyle=\tiny\color{gray},
  keywordstyle=\color{blue},
  commentstyle=\color{dkgreen},
  stringstyle=\color{mauve},
  breaklines=true,
  breakatwhitespace=true,
  tabsize=3
}


\begin{document}

\begin{titlepage}
	\begin{center}
	\begin{align*}
	\includegraphics[height=1.75in]{logo.png}
	\end{align*}


	
	\line(1,0){300}\\
	[0.25in]
	\huge{\bfseries Reflective Diary}\\
	[2mm]
	\line(1,0){200}\\
	[1.5cm]
	\textsc{\LARGE Case Study}\\
	[0.75cm]
	\textsc{\Large 4E1 Project Management}\\
	[7cm]	
	\end{center}
	
	
	
	\begin{flushright}
	\textsc{\large Alexandru Sulea\\
	D Stream\\
	\#12315152\\
	8 April 2016\\}
	\end{flushright}
	
\end{titlepage}
%Table of Contents Stuff%
\tableofcontents
%\listoffigures
%\addcontentsline{toc}{section}{List of Figures}
\listoftables
\addcontentsline{toc}{section}{List of Tables}


\thispagestyle{empty}
\cleardoublepage
\pagenumbering{arabic}
\setcounter{page}{1}

\pagebreak
\section{Simulation C}
Initially the simulation was conducted with the primary goal of delivering the product on time while maintaining the costs under budget. This solution was found to have a maximum score of 61.0 as the staff would constantly get confused and frustrated due to not knowing the situation. Outsourcing was also kept at a minimum due to the staff getting to do all the work.The stand up, peer-review and tutorial were also kept low to maximize the time the team spends working. The team size was kept at a minimum and increased to 3 only on week 4 due to the higher budget. Finishing the project on the revised deadline was not possible and the project could only be finished by week 16. The solution worked but only achieved minimal success. \\
This method was also vulnerable to low team morale and problems, if any of these problems arose the team could not perform to the best of their abilities and would finish the project late.
\\
The approach was then changed to a more rapid development cycle. This was initially a failure till the team deadline was changed to 8 weeks and the team size drastically increased from 2 to 7 team members. This timeline was chosen due to the management timeline for the simulation itself changing from 16 to 12 weeks on week 4. The team was not expected to finish the development cycle by week 8 but they would work harder thinking the deadline was shorter than it really was. Thus the team would be 3 weeks late according to the team deadline but would actually deliver the product 1 week ahead of the management deadline. To make sure that the team achieved the management deadline , the quality of the product goal was also increased to the next level, thus even if there was a problem with the team product goal, the management product goal would still be achieved.\\
The team struggled to achieve this goal initially thus outsourcing was used extensively from week one and overtime was encouraged. \\

This strategy further helped to increase the final score and have the team deliver the product ahead of schedule. \\
At this point the stand-ups and one to one meeting as well as the reviews were still being carried out on a weekly or by-weekly manner. This repeated process was found to only help at the start and would only cost time further in the development. Thus to further help the team achieve their objective the peer-review and one to one tutoring were kept to a minimum and only for the beginning of the project for 2 weeks.\\
The stand up was kept for 6 weeks due to it contributing to the teams morale and productivity.\\
The result of these changes was that the production cycle was drastically sped up but the project itself went up to 50 percent over budget. Although the project was completed over budget the fact that it was completed ahead of the revised schedule meant that it was seen as more of a success than a project finished late but under budget.





\pagebreak


\section{Simulation D}

\section{Simulation E}










	
\end{document}